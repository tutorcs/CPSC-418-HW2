\usepackage{soul,color}

% colors for stuff
\definecolor{CodeColor}{rgb}{0,0.0,0.7}
\definecolor{CommentColor}{rgb}{0.6,0.0,0.6}
\definecolor{OutColor}{rgb}{0.0,0.0,0.0}
\definecolor{URLcolor}{rgb}{0.0,0.3,0.8}
\definecolor{URLdarkcolor}{rgb}{0.0,0.0,0.6}
\definecolor{ErrorColor}{rgb}{1.0,0.0,0.0}
\definecolor{SymColor}{rgb}{0.0,0.5,0.0}
\definecolor{LightGray}{gray}{0.7}
\definecolor{LightBlue}{rgb}{0.7,0.7,1.0}
\definecolor{LightRed}{rgb}{1.0,0.7,0.7}
\definecolor{LightGreen}{rgb}{0.5,1.0,0.5}
\definecolor{LightMagenta}{rgb}{1.0,0.5,1.0}
\definecolor{MediumRed}{rgb}{0.8,0.0,0.0}
\definecolor{MediumGreen}{rgb}{0.0,0.7,0.0}
\definecolor{MediumMagenta}{rgb}{0.7,0.0,0.7}
\definecolor{DarkMagenta}{rgb}{0.5,0.0,0.5}
\definecolor{DarkGreen}{rgb}{0.0,0.5,0.0}
\definecolor{DarkRed}{rgb}{0.6,0.0,0.0}
\definecolor{DarkBlue}{rgb}{0.0,0.0,0.7}
\definecolor{anscolor}{rgb}{0.0,0.4,0.2} % for solution sets
\definecolor{red}{rgb}{0.8,0.0,0.0}
\definecolor{teal}{rgb}{0.0,0.65,0.88}

% formatting source code
\newcommand{\code}[1]{\texttt{\color{CodeColor}#1}}
\newcommand{\smallcode}[1]{\small\texttt{\color{CodeColor}#1}}
\newcommand{\footcode}[1]{\footnotesize\texttt{\color{CodeColor}#1}}
\newcommand{\scriptcode}[1]{\scriptsize\texttt{\color{CodeColor}#1}}
\newcommand{\tinycode}[1]{\tiny\texttt{\color{CodeColor}#1}}
\newcommand{\codetabs}{xx\=xx\=xx\=xx\=xx\=xx\=xx\=xx\=xx\=xx\=xx\=}
\newenvironment{xcode}{%
  \tt\color{CodeColor}\begin{tabbing}\codetabs\+\kill}{\end{tabbing}}
\newenvironment{pcode}{% pseudo-code
  \rm\color{CodeColor}\begin{tabbing}\codetabs\+\kill}{\end{tabbing}}
\newenvironment{SmallCode}{% code typeset \small
  \tt\small\color{CodeColor}\begin{tabbing}\codetabs\+\kill}{\end{tabbing}}
\newenvironment{FootCode}{% code typeset \footnotesize
  \tt\footnotesize\color{CodeColor}\begin{tabbing}\codetabs\+\kill}{\end{tabbing}}
\newenvironment{ScriptCode}{% code typeset scriptsize
  \tt\scriptsize\color{CodeColor}\begin{tabbing}\codetabs\+\kill}{\end{tabbing}}
\newenvironment{TinyCode}{% code typeset \tiny
  \tt\tiny\color{CodeColor}\begin{tabbing}\codetabs\+\kill}{\end{tabbing}}
\newcommand{\xcomment}[1]{{\color{CommentColor}\% \textrm{#1}}}
\newcommand{\ycomment}[2]{{\color{CommentColor}\% #1\textrm{: #2}}}
\newcommand{\ccomment}[1]{{\color{CommentColor}// \textrm{#1}}}
\newcommand{\xout}[1]{{\color{OutColor}#1}}
\newcommand{\xerr}[1]{{\color{ErrorColor}#1}}
\newcommand{\xsym}[1]{{\color{SymColor}\it#1}}
\newcommand{\msym}[1]{{\color{SymColor}\ensuremath{#1}}}
\newcommand{\tuple}[1]{\{#1\}}
\newcommand{\uscore}{\ensuremath{\underline{\phantom{x}}}}
\newcommand{\hs}{\rule{0.2em}{0ex}}

% the Erlang prompt:
\newcounter{erlprompt}
\renewcommand{\theerlprompt}{\arabic{erlprompt}}
\newcommand{\ereset}{\setcounter{erlprompt}{0}}
\newcommand{\estart}{\ereset%
  \xout{Erlang/OTP 20 [erts-9.0] [source] \ldots}\\%
  \xout{Eshell V9.0  (abort with $\wedge$G)}\rule{0em}{3ex}\\}
\newcommand{\epromptx}{\eprompt{}\addtocounter{erlprompt}{-1}}
\newcommand{\eprompt}{\refstepcounter{erlprompt}\theerlprompt>}

% more Erlang stuff
\newcommand{\erlist}[1]{[#1]}
\newcommand{\erlangdoc}[2]{\hrefc{http://erlang.org/doc/man/erlang.html\##1-#2}{#1}}
\newcommand{\listdoc}[2]{\hrefc{http://erlang.org/doc/man/lists.html\##1-#2}{lists:#1}}
\newcommand{\iodoc}[2]{\hrefc{http://erlang.org/doc/man/io.html\##1-#2}{lists:#1}}
\newcommand{\addpath}{\hrefc{http://erlang.org/doc/man/code.html\#add_path-1}{code:add\_path}}
\newcommand{\timeitt}{\hrefc{https://www.students.cs.ubc.ca/~cs-418/resources/erl/doc/time_it.html\#t-1}time\_it:t/1}
\newcommand{\misc}[2]{\hrefc{https://www.students.cs.ubc.ca/~cs-418/resources/erl/doc/misc.html\##1-#2}{misc:#1}}


% other handy stuff
\newcommand{\hide}[1]{}
\newcommand{\ttilde}{\texttildelow}

\newcommand{\cudadev}{\code{\_\_device\_\_}}
\newcommand{\cudaglob}{\code{\_\_global\_\_}}
\newcommand{\cudashared}{\code{\_\_shared\_\_}}

% coloring links
\usepackage{hyperref}
\newcommand{\hrefc}[2]{{\color{URLcolor}\textsf{\href{#1}{\underline{#2}}}}}
\newcommand{\hyperlinkc}[2]{{\hyperlink{#1}{\color{URLcolor}\textsf{\underline{#2}}}}}
\newcommand{\hyperlinkcd}[2]{{\hyperlink{#1}{\color{URLdarkcolor}\textsf{\underline{#2}}}}}
\newcommand{\hyperlinkcc}[2]{{\hyperlink{#1}{\code{\color{URLcolor}\underline{#2}}}}}
\newcommand{\urlc}[1]{{\color{URLcolor}\underline{\textsf{\url{#1}}}}}
\newcommand{\hyperslide}[1]{\hyperlinkc{#1}{slide \ref{#1}}}
\newcommand{\ccite}[1]{{\color{URLcolor}\cite{#1}}}

\newcommand{\bigO}{\mathcal{O}}

% copyright notice
\newcommand{\mrgcopyright}{%
  \vfill
  \begin{columns}
  \column{0.15\textwidth}
  \raggedleft
  \href{http://creativecommons.org/licenses/by/4.0/}{\includegraphics[width=0.5\textwidth]{cc-by}}\\
  \column{0.85\textwidth}
  \tiny
  Unless otherwise noted or cited, these slides are copyright 2020 by Mark Greenstreet and are\\
  made available under the terms of the Creative Commons Attribution 4.0 International license\\
  \url{http://creativecommons.org/licenses/by/4.0/}
  \end{columns}
}

% Peril-L stuff
\newcommand{\ctaglobal}[1]{\ul{\code{#1}}}

% links to Erlang documentation
\newcommand{\erldoc}[2]{\hrefc{http://erlang.org/doc/man/#1}{\texttt{#2}}}
\newcommand{\bifdoc}[2]{\hrefc{http://erlang.org/doc/man/erlang.html\##1-#2}{\texttt{#1}}}
\newcommand{\listsdoc}[2]{\hrefc{http://erlang.org/doc/man/lists.html\##1-#2}{\texttt{#1}}}
\newcommand{\miscdoc}[2]{\hrefc{http://www.ugrad.cs.ubc.ca/~cs418/resources/erl/doc/misc.html\##1-#2}{\texttt{misc:#1}}}
\newcommand{\timeitdoc}[2]{\hrefc{http://www.ugrad.cs.ubc.ca/~cs418/resources/erl/doc/time\_it.html\##1-#2}{\texttt{time\_it:#1}}}
\newcommand{\workersdoc}[2]{\hrefc{http://www.ugrad.cs.ubc.ca/~cs418/resources/erl/doc/workers.html\##1-#2}{\texttt{workers:#1}}}
\newcommand{\wtreedoc}[2]{\hrefc{http://www.ugrad.cs.ubc.ca/~cs418/resources/erl/doc/wtree.html\##1-#2}{\texttt{wtree:#1}}}


% other handy stuff
\newcommand{\squote}{\textquotesingle}
\newcommand{\mrgemph}[1]{{\color{DarkMagenta}\textbf{#1}}}
\newcommand{\lyse}{\hrefc{http://learnyousomeerlang.com}{\textit{Learn You Some Erlang}}}
\newcommand{\LYSE}{\hrefc{http://learnyousomeerlang.com}{\textit{LYSE}}}
\newcommand{\pacheco}{\hrefc{http://resolve.library.ubc.ca/cgi-bin/catsearch?bid=10297705}{\textit{Intro. to Parallel Computing (Pacheco)}}}


